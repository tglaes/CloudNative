%*******************************************************
% Abstract
%*******************************************************
\pdfbookmark[0]{Zusammenfassung}{Zusammenfassung}
\chapter*{Zusammenfassung}
Diese Ausarbeitung beschäftigt sich mit dem Thema Cloud-Native Architektur. Einleitend wird auf das Cloud Computing eingegangen, welches die Basis für Cloud-Native Architekturen bildet. Anschließend wird Cloud-Native als eine Technologien definiert, die es Unternehmen ermöglicht moderne und dynamischen Umgebungen zu implementieren und zu betreiben.\\
Als nächstes wird die Microservice-Architektur genauer erläutert. Microservices werden auf Grund des Architekturstils gerne in Cloud-Native Anwendungen verwendet und sind somit ein wichtiger Bestandteil in Cloud-Native Systemen. Des Weiteren erfolgt eine Gegenüberstellung der Microservices mit der monolithischen Architektur, wodurch sich Unterschiede herausfiltern lassen.\\
Auch die Containerisierung trägt eine bedeutende Rolle in der Cloud-Native Architektur. Durch Container können Anwendungen von ihrer Ausführungsumgebung abstrahiert werden, sodass diese schnell und zuverlässig bereitgestellt werden können. Containerbasierte Anwendungen werden außerdem von virtuellen Maschinen abgegrenzt.\\
Die Seminararbeit befasst sich zudem mit den wichtigsten Eigenschaften der Cloud-Native Architektur, wodurch sich die Vor- und Nachteile dieser Architektur herauskristallisieren. Dabei wird auch deutlich, dass sich die Eigenschaften der Microservices in Cloud-Native widerspiegeln. Beispiele dafür wäre zum Einen die Skalierbarkeit und Flexibilität und zum Anderen die Änderbarkeit und Wartbarkeit.\\
Das Cloud-Native Kapitel wird mit typischen Einsatzgebieten abgeschlossen. Diese stellen meist Systeme dar, welche ein hohes Maß an Skalierbarkeit erfordern. Cloud-Native Architekturen werden besonders in den Bereichen Streaming und Big Data verwendet.\\
Ein weiteres großes Kapitel befasst sich mit einem selbst entworfenem Anwendungsfall einer Cloud-Native Architektur, welcher auch prototypisch umgesetzt wurde. Dabei werden die funktionalen sowie nicht-funktionalen Anforderungen beschrieben. Anhand eines Architekturentwurfs wird der Aufbau der Anwendung deutlich und die einzelnen Bestandteile der Architektur werden genauer erklärt.\\
Durch das Sequenzdiagramm wird die Funktionsweise der Anwendung bzw. die Beziehungen der einzelnen Services untereinander graphisch dargestellt.\\
Das letzte Kapitel thematisiert die Diskussion des entworfenen Prototyps. In diesem wird überprüft, ob die Anwendung einer Cloud-Native Architektur entspricht. Dabei werden Schwachstellen und Verbesserungsmöglichkeiten aufgezeigt und abschließend ein Fazit gezogen.