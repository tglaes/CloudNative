%=========================================
% 	   Cloud Native     		 =
%=========================================
\chapter{Cloud Native}

In diesem Kapitel gehen wir auf die Definition von Cloud-Native Architekturen ein, grenzen sie von anderen Absätzen ab und betrachten wichtige Eigenschaften und Design-Prinzipien. Abschließend beschäftigen wir uns mit den Vor-und Nachteilen und den typischen Einsatzgebieten.

\section{Cloud Computing}
Bevor wir uns mit Cloud-Native auseinandersetzten können, müssen wir uns zuerst mit dem Cloud Computing beschäftigen, da es nämlich die Basis für diese Architekturen bildet und sie maßgeblich beeinflusst. Die NIST Definition von Cloud Computing enthält die wichtigsten Merkmalen.

Cloud computing is a model for enabling ubiquitous, convenient, on-demand network access to a shared pool of configurable computing resources (e.g., networks, servers, storage, applications, and services) that can be rapidly provisioned and released with minimal management effort or service provider interaction. This cloud model is composed of five essential characteristics, three service models, and four deployment models. TODO

Entscheidend für Cloud-Native ist nun die schnelle Bereitstellung von Resourcen, denn dies eröffnet neue Möglichkeiten hinsichitlich der Skalierbarkeit und hat dadurch einen starken Einfluss auf die Architekturen.

\section{Definition Cloud Native}
Was genau Cloud-Native ist und wie man es definieren kann ist schwierig, da der Begriff noch relativ jung ist. Eine erste Version einer Definition kommt von der Cloud Native Computing Foundation, einer Organisation, die als Vorreiter in Sachen Cloud-Native gilt.

CNCF Cloud Native Definition v1.0

Cloud native Technologien ermöglichen es Unternehmen, skalierbare Anwendungen in modernen, dynamischen Umgebungen zu implementieren und zu betreiben. Dies können öffentliche, private und Hybrid-Clouds sein. Best Practices, wie Container, Service-Meshs, Microservices, immutable Infrastruktur und deklarative APIs, unterstützen diesen Ansatz.

Die zugrundeliegenden Techniken ermöglichen die Umsetzung von entkoppelten Systemen, die belastbar, handhabbar und beobachtbar sind. Kombiniert mit einer robusten Automatisierung können Softwareentwickler mit geringem Aufwand flexibel und schnell auf Änderungen reagieren.

Die Cloud Native Computing Foundation fördert die Akzeptanz dieser Paradigmen durch die Ausgestaltung eines Open Source Ökosystems aus herstellerneutralen Projekten. Wir demokratisieren modernste und innovative Softwareentwicklungs-Patterns, um diese Innovationen für alle zugänglich zu machen.

TODO


Esch fragen, ob man noch auf, Service-Meshs, immutable Infrastruktur und deklarative APIs


\section{Microservices}
definition
warum braucht man das im cloud native bereich
Eigenschaften, Vorteile
- um diese microservices zu verwalten werden ...
\section{Container}
definition
warum braucht man das im cloud native bereich
Eigenschaften, Vorteile
 Docker, Kubernetes
\section{Abgrenzungen}
gegenüber monolith
Definition, kurz: Eigenschaften anhand von Bild
gegenüber virtualisierung
kurz: Definition, Vergleich/Abgrenzung anhand von Bild
\section{Eigenschaften}
Skalierbarkeit/Elastizität
Infrastucture is fluid
Updates und Tests
Sicherheit
Globale Ebene
\section{Design Prinzipien}
twelve factor app
\section{Vor- und Nachteile}
die von der Powerpoint
\section{Einsatzgebiete}
vielleicht in einleitung