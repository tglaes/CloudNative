%=========================================
% 	   Fazit     		 =
%=========================================
\chapter{Fazit}
Das Ziel der Seminararbeit war es, sich mit dem Thema Cloud-Native Architekturen auseinander zu setzten und prototypisch eine solche Architektur zu implementieren.\\
In den Kapiteln eins und zwei wurde das Thema Cloud-Native Architekturen genauer betrachtet und unter anderem die Definition, Eigenschaften und Vor-und Nachteile dargestellt. Bereits in diesen einleitenden Kapiteln wird die Komplexität des Cloud-Native Bereiches deutlich. Auf Grund der Neuheit und der aktuellen Präsenz des gesamten Cloud-Bereichs, gibt es viele unseriöse Quellen, die auf nicht wissenschaftlichem Halbwissen basieren. Dies erschwert die Suche nach zuverlässigen Informationen, welche herausgefiltert werden müssen.\\
In den Kapiteln drei und vier wird ein eigens formuliertes Fallbeispiel beschrieben und die entworfene Architektur sowie die prototypische Implementierung vorgestellt und abschließend diskutiert. Als Ergebnis resultiert eine Microservice Architektur, die mit einem API-Gateway und einer beispielhaften graphischen Oberfläche ergänzt wurde. Beim Entwurf der Architektur und bei der anschließenden Implementierung waren schon Schwachstellen zu erkennen, die in Kapitel vier nochmal aufgegriffen wurden. Ein Beispiel für eine solche Schwachstelle ist die Skalierbarkeit. Der Message-Service und der Login-Service können in dem entwickelten Prototypen nicht unabhängig von einander skaliert werden. Für dieses Problem wurde auch eine Lösung aufgezeigt, die jedoch nicht mehr im Rahmen dieser Ausarbeitung umgesetzt werden konnte.\\
Dennoch wurde ein funktionsfähiger Prototyp nach einer Microservice-Architektur implementiert, welcher die Eigenschaften des Cloud-Native Bereichs widerspiegelt.\\
\\
Abschließend lässt sich sagen, dass Cloud-Native Architekturen ein sehr interessantes aber auch mächtiges Werkzeug sind, um mit sehr großen Menge von Daten und Benutzern umzugehen. Jedoch sind sie sehr komplex und benötigen eine gewisse Expertise und Erfahrung um sie effektiv einzusetzen. Daher werden sie auch hauptsächlich von Unternehmen wie Google, Facebook und Netflix verwendet.\\
Cloud-Native befindet sich noch in einer jungen Entwicklungsphase, spielt aber bereits jetzt schon eine wichtige Rolle im Cloud-Bereich. Cloud-Native wird sich im Laufe der Jahren immer weiter entwickeln. Dadurch könnte sich die Komplexität immer mehr verringern (z.B. durch Frameworks und andere Tools), sodass Cloud-Native Architekturen auch für kleinere Vorhaben eine Rolle spielen.
