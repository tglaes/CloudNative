%=========================================
% 	   Fazit     		 =
%=========================================
\chapter{Fazit}

In diesem Kapitel ziehen wir ein Fazit aus der Seminararbeit.\\
\\
Ziel war es, sich mit dem Thema Cloud-Native Architekturen auseinander zu setzten und prototypisch eine solche Architektur zu implementieren.\\
In den Kapiteln eins und zwei haben wir das Thema Cloud-Native Architekturen genauer betrachtet und uns unter anderem die Definition, Eigenschaften sowie Vor-und Nachteile genauer angesehen. Dabei ist uns bereits aufgefallen, dass der Cloud-Native Bereich komplex ist. Da er auch noch sehr neu ist, war es nicht einfach zuverlässige Informationen zu finden. Hinzu kommt, dass der gesamte Cloud Bereich momentan sehr angesagt ist, sodass man sich durch eine Menge von unseriösen, unzuverlässigen Informationen durcharbeiten musste. Dennoch konnten wir die wichtigsten Punkte des Cloud-Native Bereiches identifizieren und in dieser Arbeit festhalten.\\
In den Kapiteln drei und vier haben wir unser Fallbeispiel beschrieben, die entworfene Architektur sowie die prototypische Implementierung vorgestellt und abschließend diskutiert. Das Ergebnis war eine Microservice Architektur, die wir mit einem API-Gateway und einer beispielhaften grafischen Oberfläche ergänzt haben. Beim Entwurf der Architektur und bei der anschließenden Implementierung waren schon erste Schwachstellen zu erkennen, die wir dann in Kapitel vier nochmal aufgegriffen haben. Dazu erläuterten wir einige dieser Schwachstellen genauer und schlugen Lösungen vor.\\
Abschließend lässt sich sagen, dass Cloud-Native Architekturen ein mächtiges Werkzeug sind um mit der sehr großen Mengen von Daten und Benutzern umzugehen. Jedoch sind sie sehr komplex und benötigen eine gewisse Expertise und Erfahrung um sie effektiv einzusetzen. Daher werden sie auch hauptsachlich von Unternehmen wie Google, Facebook und Netflix verwendet. Da der Bereich ständig weiterentwickelt wird es ist aber denkbar, dass man es schafft diese Komplexität zu verringern (z.B. durch Frameworks und andere Tools), sodass Cloud-Native Architekturen auch für kleinere Vorhaben eine Rolle spielen.