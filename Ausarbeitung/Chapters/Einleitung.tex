%=========================================
% 	   Einleitung     		 =
%=========================================
\chapter{Einleitung}
Cloud-Dienste sind heutzutage nicht mehr aus unserem Alltag wegzudenken. Auch im Arbeitsleben bzw. für viele Unternehmen ist das Thema Cloud im Zusammenhang mit der Digitalisierung ein wichtiger Bestandteil geworden. Dadurch wurden neue Geschäftsmodelle ermöglicht und gleichzeitig die Wettbewerbsfähigkeit des Unternehmens gesteigert.\\
\\
Ein neuer Ansatz ist Cloud-Native. Hierbei werden die Applikationen von Beginn an für den Einsatz in der Cloud entwickelt. Dieser wird als Zukunft der Software-Entwicklung gesehen und basiert auf den bewährten Technologien des Cloud Computings.\\
Die Weiterentwicklung und der Ausbau des Cloud Computing Bereiches, welche durch die steigenden Zahlen von Benutzern und Daten schnell vorangetrieben werden, benötigen auch entsprechende Softwaresysteme. Diese können mit den bereitgestellten Ressourcen effektiv umgehen und sind auf die Cloud Umgebung abgestimmt.\\
Cloud-Native hat sich in den letzten Jahren als eigener Bereich in der Informatik herauskristallisiert und wurde vorallem durch große Unternehmen wie Netflix, Google und Amazon geprägt.\\
Auch die Verbreitung von Cloud Native Technologien insbesondere der Container und Container-Orchestrierungs-Tools wie z.B. Kubernetes sind in den letzten Monate deutlich gestiegen.\\
Durch Cloud-Native ergibt sich eine neue Möglichkeit große komplexe Systeme zu entwickeln. Dies ist besonders für Unternehmen interessant, da somit in kürzester Zeit innovative Anwendungen entwickelt werden können.\\
Benutzer bzw. Kunden werden zudem immer anspruchsvoller und erwarten, dass schnellstmöglich neue Anwendungen für sie zur Verfügung stehen. Des Weiteren wird eine schnelle Reaktionsfähigkeit, innovative Features und eine möglichst geringe Ausfallzeit erwartet.\\
Der Cloud-Native Ansatz ist für solche Situationen ausgelegt und bietet die Möglichkeit Applikationen, je nach Bedarf zu skalieren.\\
\\

https://www.cncf.io/blog/2020/08/14/state-of-cloud-native-development/