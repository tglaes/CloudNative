%=========================================
% 	   Diskussion     		 =
%=========================================
\chapter{Diskussion}
In diesem Kapitel diskutieren wird die entworfene Architektur und den dazugehörigen Prototypen.

\section{Vergleich mit Eigenschaften Cloud Nativ}
In diesem Abschnitt sehen wir uns an, welche Cloud-Native Eigenschaften unsere Architektur erfüllt und welche nicht.\\
\\
1.) Skalierbarkeit\\
Die Kombination von Microservices und eines Load-Balancers (API-Gateway) erfüllt auf den ersten Blick die Eigenschaft, denn es können Dank des Designs des Message-Microservices mehrere Instanzen gleichzeitig verwendet werden, auf die dann der Load-Balancer die Anfragen gleichmäßig verteilt. Da jedoch der Message-Microservice bei jeder Anfrage überprüfen muss, ob der vom Benutzer bereitgestellte Token gültig ist, wird für jede Anfrage auch eine weitere Anfrage an den Login-Microservice generiert. Diese Abhängigkeit wirkt sich stark negativ auf die Skalierbarkeit aus, da der Effekt von mehreren Message-Microservice Instanzen gleich null ist, wenn nur eine Instanz des Login-Microservices vorhanden ist. Dieses Problem ist Lösbar, indem man ebenfalls vom Login-Microservice mehrere Instanzen laufen lässt und die Anfragen des Message-Microservices an das API-Gateway sendet, welches dann die Anfragen an die Login-Microservice Instanzen verteilt.\\
\\
2.) Aufgebaut auf der Annahme \glqq infrastructure is fluid and failure is constant\grqq{}\\
Hier ist leicht festzustellen, dass die Architektur diese Eigenschaft nicht erfüllt. Es wird nicht berücksichtigt, dass Infrastruktur sich ändern kann oder das Fehler auftreten können. Beide dieser Eigenschaften sind essentiell für Cloud-Native Architekturen, da ohne diese keine Anwendung in einer Cloud-Umgebung bestehen kann. Die Architektur hat jedoch das Potenzial diese Eigenschaften zu erfüllen, indem man z.B. Docker in Kombination mit Kubernetes verwendet.\\
\\
3.) Updates und Tests verlaufen unscheinbar\\
Wir können eine neue Version des Systems (API-Gateway und Mircoservices) auf neuen Strukturen installieren und testen, während die alte Version weiter verfügbar ist. Sind Installation und Test abgeschlossen kann der Verkehr von den Clients auf die neue Version geleitet werden. Dieser Ansatz nennt sich Immutable Infrastructure und wäre eine Möglichkeit diese Eigenschaft umzusetzten. Durch den modularen Aufbau der Architekur, können auch einzelne Microservices auf gleiche Weise upgedated werden. Insgesamt erfüllt die Architektur diese Eigenschaft in der Theorie wurde aber nicht in der Praxis ausgiebig getestet.\\
\\
4.) Sicherheit ist ein Teil der Architektur\\
Diese Eigenschaft ist erfüllt. Einerseits ist ein Authorisierungsmechanismus mithilfe von Tokens vorhanden, andererseits können im API-Gateway z.B. Logging und Schutz vor DDoS-Attacken implementiert werden. Es sei gesagt, dass dadurch das System noch weit entfernt ist von einem in der Praxis sicherem System.\\
\\
5.) Globale Ebene\\
Die Architektur ist nicht für eine globale Ebene geeignet. Es wäre zwar möglich die Anwendung mehrfach zu installieren, jedoch würden diese Installationen keine Daten teilen. Um dies zu bewerkstelligen müsste man auf verteilte Datenbanken bauen.

\section{Microservices}
 Docker, Kubernetes
 
\section{Tradeoff}
\label{tradeoff}
Schon beim Entwurf der Architektur fällt auf, dass der Message-Microservice abhängig vom Login-Microservice ist, da er jede Anfrage authorisieren muss. Skaliert man nun den Message-Mircoservice muss man auch den Login-Microservice hochskalieren, denn ansonsten wird der Login-Microservice zu einem Flaschenhals und man hat keine skalierbares System. Generell wird an der Architektur der Zielkonflikt zwischen Sicherheit und Sklierbarkeit deutlich, da die eingesetzten Sicherheitsmaßnahmen (API-Gateway und Login-Microservice) das System verlangsamen. Jede Anfrage durchläuft eine Sicherheitsüberprüfungen im API-Gateway und muss danach noch von dem jeweiligen Microservice beim Login-Microservice authentifiziert werden.


\section{TODO}
Will sich ein Benutzer registrieren so wird eine Anfrage an den Registration-Microservice generiert, welcher den neuen Benutzer dann anlegt und danach eine Anfrage an den Login-Microservice sendet, damit dort die Kredentialien gespeichert werden. Tritt nun ein Fehler auf nachdem der Benutzer in der Registration-Microservice Datenbank gespeichert wurde, wird kein Eintrag in der Login-Microservice Datenbank erzeugt, sodass der Benutzer registriert ist sich aber nicht einloggen kann. Solch eine Situation ist natürlich zu vermeiden. Es soll an diesem Beispiel deutlich werden, dass nicht nur Orchestrierung für Robustheit und Widerstandsfähigkeit verantwortlich sind, sondern auch jeder Microservice muss dafür Sorge tragen. Eine Lösung in unserem Fall wäre nach dem Ausfall des Login-Microservices einen Rollback zu machen und einen Fehler auszugeben. Ein weiterer Lösungansatz ist, eine andere Instanz des Login-Microservices zu verwenden.\\
\\
TODO Security

\section{Erweiterbarkeit}
Auf Grund der Microservice-Architektur lässt sich die Anwendung gut erweitern. Dies ist der Fall, da die Anwendung aus verschiedenen Microservices besteht, die größtenteils unabhängig mit einander fungieren.\\
Wie in Abschnitt \ref{tradeoff} beschrieben, muss bei einer Skalierung des Message-Services auch der Login-Service berücksichtigt werden. Was die Erweiterbarkeit in diesem Fall einschränkt.\\
Die restlichen Microservices wie z.B. das API-Gateway und die Benutzeroberfläche laufen unabhängig von einander. Diese können erweitert werden, ohne die Funktionalität der anderen Services zu beeinträchtigen.
TODO sonst unabhängig??? TODO